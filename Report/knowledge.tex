\section{Blockchain}
La blockchain è una tecnologia che si basa sul concetto di Distributed Ledger Technology (DLT), ossia un registro distrbuito digitale. Alcune caratteristiche fondamentali:
\begin{itemize}
  \item {\bfseries decentralizzazione}: non c'è un single-point of failure, non ci si affida a servizi proprietari, essa è composta da nodi e chiunque può mettere in piedi diversi nodi molto semplicemente (per questo si legge spesso che la blockchain è "democratica")
  \item {\bfseries trasparenza}: chiunque ha accesso ad ogni transizione mai registrata in questo registro distribuito
  \item {\bfseries sicurezza}: ogni informazione ha un hash associato, questo permette di validare i dati ogni volta è necessario
	\item {\bfseries immutabilità}: è molto complesso modificare una transizione salvata sulla blockchain (è necessario che la maggioranza dei nodi accettino questa modifica)
\end{itemize}

Letteralmente blockchain significa "catena di blocchi", infatti questo registro distribuito è composto concatenando una serie di blocchi, ovvero "pacchetti" di transizioni (e altri dati necessari a costruire una catena).

\subsection{Blocchi}
Questo 'libro mastro' cresce accodando nuovi blocchi (o record), ognuno dei quali fa riferimento al blocco precedente attraverso un hash.
Ogni blocco, oltre che contenere una lista di transizioni, ha anche un header in cui sono presenti campi di gestione del blocco stesso.
Alcuni dei campi più importanti contenuti nell' header sono:
\begin{itemize}
    \item {\bfseries Hash}: l'hash con cui identifichiamo il blocco
	\item {\bfseries PrevHash}: è un hash che serve per fare riferimento al precedente blocco della catena
	\item {\bfseries Merkle root}: hash di tutti gli hash di tutte le transazioni nel blocco
	\item {\bfseries Timestamp}: rappresenta il time stamp dell'ultima transazione con un algoritmo conosciuto come Unix hex timestamp
	\item {\bfseries Numero di transazioni}: indica il numero di transizioni contenute nel blocco
\end{itemize}

\subsection{Transizioni}
La transizione è forse l'entità più interessante della blockchain. Infatti essa definisce il trasferimento di informazioni da un determinato account ad un altro. Ad esempio nei bitcoin è il passaggio di valuta da un account ad un altro; in questo caso la transizione dovrà contenere i riferimenti agli account e la quantità di bitcoin scambiata.


\section{Ethereum}

\begin{center}
    \includegraphics[scale=0.3]{ethereum.jpg}
\end{center}

Ethereum è una piattaforma che permette a chiunque di costruire applicazioni decentralizzate eseguite sulla blockchain.
\'E un progetto open-source, questo significa che nessuno è proprietario di Ethereum e che è stato sviluppato da persone di tutto il mondo.

\vspace{0.5cm}

Si può dire che Ethereum è una piattaforma perchè è stato progettato per essere flessibile e per adattarsi a tutti i contesti, infatti chiunque può fargli eseguire una propria applicazione, uno Smart Contract.
Alla base di questa flessibilità c'è la EVM (Ethereum virtual machine) che è il componente che si occupa di eseguire gli smart contracts.

\vspace{0.5cm}

Gli smart contracts sono dei pezzi di codice salvati sulla blockchain e che vengono eseguiti a richiesta.
Solidity è il linguaggio con cui è possibile implementarli, viene definito un linguaggio "contract-oriented".
Dal punto di vista tecnico i contratti non sono altro che account: in Ethereum ci sono due tipi di account, gli EOAs (Externally owned accounts) che sono account associati a persone ed i Contract Account che sono governati dal loro codice interno ma debbono essere attivati da un EOA.
Ogni account ha un saldo associato in ETH (la moneta ufficiale di Ethereum); nel caso di EOA il saldo viene aggiornato ad ogni transizione in entrata o in uscita, mentre nel caso dei Contract Account ad essi viene addebitata un costo per ogni volta che vengono eseguiti (si paga un costo computazionale).

\vspace{0.5cm}

Altra carta vincente di Ethereum sono le DAPP: sostanzialmente sono delle semplici applicazioni (mobile, web, desktop ecc) che interagendo con gli smart contracts vanno ad aggiungere transizioni sulla blockchain.
Praticamente queste applicazioni usano la blockchain come un "server" decentralizzato con tutti i vantaggi conseguenti.

\vspace{0.5cm}

Creare un nodo di Ethereum è molto semplice, è sufficiente installare uno dei suoi client ed aspettare che si sincronizzi. Ci sono numerosi client, tra i più famosi: Parity, go-ethereum, cpp-ethereum, pyethapp, ruby-ethereum...
Questi client mettono a disposizone diverse funzionalità tra cui l'accesso alla blockcahin, esponendo delle api da richiamare per leggere i dati della blockchain, la gestione delle DAPP ed il mining.
Il mining consiste nella possibilità di mettere a disposizione la propria potenza computazionale per la blockchain: praticamente il nodo miner si occupa di ricevere, propagare, validare ed eseguire transizioni, di raggrupparle in blocchi e poi compete con altri miner per fare in modo che il prossimo blocco della blockchain sia quello appena creato. Il miner viene compensato con degli ETH per ogni blocco aggiunto.

\section{Web3.js}
Web3.js è una libreria Javascript che espone delle semplici api che ci permettono di interagire con Ethereum.
Internamente la libreria non fa altro che mappare le proprie funzioni con le api esposte da un nodo Ethereum effettuando le relative chiamate RPC.
Può far riferimento a nodi sia locali che remoti, è sufficiente cambiare opportunamente l'url in cui è esposto il nodo.

Progetto open-source disponibile su GitHub:
https://github.com/ethereum/web3.js/


\section{Angular}

Angular è un framework per lo sviluppo di applicazioni, originariamente pensato per il web, si è poi esteso anche al mondo mobile e desktop.
Anche questo è un progetto open-source al cui sviluppo/mantenimento partecipa in maniera molto importante, tra gli altri sviluppatori e aziende, il Team Angular di Google.

Angular rende più semplice creare applicazioni e lo fa combinando l'approccio dichiarativo per la definizione della UI a pattern tipici dei linguaggi object-oriented come la dependency-injection.
A tutto questo aggiunge una serie di componenti e servizi già implementati come ad esempio tutti i componenti grafici che seguono le linee guida del Material Design di Google o il servizio http, molto utile per l'interazione con delle api di backend.

\subsection{Typescript}
Il linguaggio usato nello sviluppo con Angular è Typescript. Il typescript è un linguaggio compilato superset del javascript, il risultato della sua compilazione è proprio codice javascript minificato e che segue tutte le best practices.
Quindi praticamente la nostra applicazione Angular al momento dell'esecuzione altro non è che un'enorme applicazione javascript.

L'uso del typescript mette a disposizione dello sviluppatore tutta una serie di utility tipiche dei linguaggi object-orented: classi, interfacce, ereditarietà...
Permette quindi di ottenere applicazioni complesse con codice ben strutturato.

\subsection{UI}
In angular la ui è dichiarativa e si definisce usando l'html con l'aggiunta di alcuni costrutti e pattern specifici del framework come ad esempio il {\bfseries Data Binding}.
Il Data binding ci permette di creare un legame tra un componente grafico ed una nostra classe scritta in Typescript.
Questo approccio è molto scalabile e permette di risparmiare molto codice in quanto è possibile modificare quanto mostrato a schermo soltanto modificando una variabile in memoria.
Affinchè questo possa accadere angular effettua dei cicli di re-rendering alla modifica di determinate variabili.
Per quanto riguarda la logica di iterfaccia del nostro applicativo è definita nei {\bfseries Componenti}, che sono classi a cui applichiamo il decoratore {\bfseries Component}.

\subsection{Business logic}
La logica di business è implementata nei {\bfseries Servizi}, ovvero classi typescript che usano il decoratore {\bfseries Injectable} di Angular.
Come definito dal decoratore injectable i servizi possono essere iniettati attraverso la {\bfseries Dependency Injection}.
L'applicazione angular è organizzata in moduli ed almeno un modulo root deve esistere. Ogni modulo importa ed esporta una serie di componenti e servizi (oltre ad altri strumenti secondari).

\subsection{Angular/RxJs}
Angular ha aggiunto nelle ultime versioni il pacchetto RxJs: questa libreria permette l'uso del "Reactive programming", ovvero di un paradigma di programmazione asincrona che si occupa dello stream e propagazione di dati.
