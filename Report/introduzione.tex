\par \noindent
Nell'ambito del Project ci siamo occupati della realizzazione di un tool per l'interrogazione e l'integrazione di dati residenti sulla blockchain Ethereum.
\vspace{0.5cm}
\par \noindent
Il primo passo è stato quello di costruirsi un solido background teorico: siamo partiti dallo studio della blockchain in generale approfondendo i suoi principi e le sue caratteristiche. Siamo poi passati ad Ethereum cercando di capire le differenze con le altre piattaforme disponibili nel mercato, con relativi vantaggi e svantaggi.
\vspace{0.5cm}
\par \noindent
Nello step successivo ci siamo dedicati alla parte più tecnica, abbiamo scelto Angular come framework di sviluppo e la libreria web3.js per l'interazione con la blockchain. Durante lo sviluppo è risultato fondamentale l'uso di un approccio "funzionale" nella strutturazione del progetto e questa necessità ci ha portato ad usare anche la libreria rxjs.

\vspace{0.5cm}

\par \noindent
Il repository del progetto è disponibile al link:

https://github.com/MaxUnicam/QueryingEthereum