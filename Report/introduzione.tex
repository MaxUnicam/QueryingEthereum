\par \noindent
Nell'ambito del Project ci siamo occupati della realizzazione di un tool per l'interrogazione e l'integrazione di dati residenti sulla blockchain Ethereum.
\vspace{0.5cm}
\par \noindent
Il primo passo è stato quello di costruirsi un solido background teorico: siamo partiti dallo studio dal concetto di blockchain approfondendo i suoi principi e le sue caratteristiche. Siamo poi passati ad Ethereum cercando di capire le differenze con le altre piattaforme disponibili sul mercato, con relativi vantaggi e svantaggi.
\vspace{0.5cm}
\par \noindent
Nello step successivo ci siamo dedicati alla parte più tecnica, abbiamo scelto Angular come framework di sviluppo, la libreria web3.js per l'accesso ai dati della blockchain e la libereria rxjs per la costruzione di uno stream di dati che viene filtrato progressivamente durante la query.

\vspace{0.5cm}

\par \noindent
Il repository del progetto è disponibile al link:

\url{https://github.com/MaxUnicam/QueryingEthereum}
