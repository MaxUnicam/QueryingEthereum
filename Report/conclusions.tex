\section{Sviluppi futuri}
Questo progetto può essere migliorato sotto diversi aspetti.
Ad esempio nella costruzione della query si possono concatenare diversi Constraint però l'ordine in cui vengono eseguiti è lo stesso in cui vengono aggiunti.
Pur essendo accettabile sarebbe sicuramente meglio costruire query complesse in cui l'ordine possa essere definito anche dall'utente attraverso l'uso delle parentesi.
Questa modifica dal punto di vista tecnico è piuttosto semplice, è sufficiente costruire un albero invece di una lista, più complessa invece è la resa grafica, ovvero trovare un modo di aggiungere parentesi in modo arbitrario che sia anche semplice ed intuitivo.

\vspace{0.5cm}

Altra modifica interessante potrebbe essere quella di aggiungere una gestione degli utenti, questo permetterebbe di mantenere uno storico di query eseguite, i loro risultati, preferenze che durano per diverse sessioni ecc.

\vspace{0.5cm}
Bisognerebbe indagare e studiare in modo approfondito i Web Workers che sono strumenti che dovrebbero permettere il multithreading in Javascript.
Potrebbero essere un ulteriore miglioramento alle prestazioni del tool sviluppato.

\vspace{0.5cm}
Altro punto in cui sarebbe interessante lavorare è la gestione di nuove query che al momento non sono state implementate:
count, max, min, first ecc.

\section{Conclusioni}
Il progetto è risultato piuttosto 'challenging' sia per la necessità di acquisire competenze teoriche sui temi della blockchain e sia dal punto di vista tecnico,
non tanto per l'uso di Angular ma per la necessità di trovare una soluzione a problemi che erano conseguenza della natura delle tecnologie che abbiamo usato.

\vspace{0.5cm}

Proprio per questo è stato ancora più interessante del previsto!